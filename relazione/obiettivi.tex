\section{Metodi di tracking basati su modelli}
Il \textit{video tracking} è il processo secondo il quale si localizza un oggetto in movimento all'interno di uno stram video e rappresenta uno dei più interessanti problemi di \textit{computer vision}. Esistono svariati approcci al video tracking, ognuno orientato ad ottimizzare le prestazioni relativamente al campo d'azione. In questo lavoro è stata effettuata la scelta di effettuare il tracking secondo l'approccio basato su modelli, che viene eseguito secondo due passi fondamentali: la localizzazione dell'oggetto da tracciare ed il tracciamento effettivo.

Il primo passo, computazionalmente non molto oneroso, è stato realizzato tramite il \textit{background subtraction} (descritto nella sezione ???) e consiste nella rilevazione dell'oggetto all'interno dell'immagine e nell'ottenimento delle informazioni relative.

Il secondo passo, ovvero l'applicazione del tracking al video, rappresenta il punto di maggior interesse del lavoro in quanto consiste nell'elaborazione di una stima della posizione al frame successivo dell'oggetto selezionato; l'esecuzione si basa sull'elaborazione dei dati ottenuti dal processo di localizzazione dell'oggetto.

Gli obiettivi di questo elaborato sono la realizzazione, l'analisi e la comparazione dei due più importanti algoritmi di tracking basato su modelli: il filtro di Kalman, descritto nella sezione \ref{kalman} ed il Condensation, descritto nella sezione \ref{condensation}.

\textbf{Questa parte va espansa mettendo un po' di discorsi sul tracking generale e poi andando a parare sul tracking model based}

\subsection{Kalman Filter}\label{kalman}
Il Kalman Filter\cite{kalman-intro} è un efficiente filtro ricorsivo che valuta e stima lo stato di un sistema dinamico sulla base di una serie di misure soggette a rumore. Il filtro è molto potente in quanto supporta la stima degli stati passati, presenti e futuri del sistema anche quando la natura del sistema è sconosciuta. \'E usato in molti campi ingegneristici, che vanno dall'applicazione in tecnologie radar all'applicazione in computer vision, come utilizzato in questo stesso ambito.

Per interpretare correttamente la rappresentazione matematica del filtro di Kalman è necessario introdurre i componenti del framework di lavoro, ovvero le matrici $A_k, B_k, F_k, H_k, P_k ,Q_k, R_k$

\textbf{descrivere seriamente il kalman filter (tipo dagli articoli o dalla wiki) mettendo anche dettagli matematicosi}

\subsection{ConDenSation}\label{condensation}
\textbf{descrivere seriamente il particle filter (tipo dagli articoli o dalla wiki) mettendo anche dettagli matematicosi}

\subsection{Descrizione dell' implementazione dei modelli}
Fare cappello su cosa è un modello poi particolareggiare verso il nostro modello

QUESTA PARTE VA FATTA BENE

ricordarsi di mettere la descrizione del background subtraction per quanto riguarda il ruolo dell'oggetto.

come sono fatte le matrici che descrivono 
dire che noi s'ha un generico sistema descritto unicamente dalla sua posizione sul piano e dalla sua velocità orizz e verticale

Cosa si prende per varianza di uno dell'altro, come è fatto lo stato (vettore i 4 dimensioni di cui 2 posizione xy etc..)
