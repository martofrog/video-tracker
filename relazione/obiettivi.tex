\chapter{Metodi di tracking basati su modelli}
Effettuare il tracking di oggetti in sequenze di immagini è attualmente uno dei più interessanti problemi di \textit{computer vision}. In questo lavoro è stato effettuato il confronto tra due tipologie diverse di tracking basate su modelli: il filtro di Kalman \cite{kalman-intro} ed il Condensation \cite{condensation}.

Questa parte va espansa mettendo un po' di discorsi sul tracking generale e poi andando a parare sul tracking model based

\section{Kalman Filter}
descrivere seriamente il kalman filter (tipo dagli articoli o dalla wiki) mettendo anche dettagli matematicosi

\section{ConDenSation}
descrivere seriamente il particle filter (tipo dagli articoli o dalla wiki) mettendo anche dettagli matematicosi

\section{Descrizione dell' implementazione dei modelli}
Fare cappello su cosa è un modello poi particolareggiare verso il nostro modello

QUESTA PARTE VA FATTA BENE

ricordarsi di mettere la descrizione del background subtraction per quanto riguarda il ruolo dell'oggetto.

come sono fatte le matrici che descrivono 
dire che noi s'ha un generico sistema descritto unicamente dalla sua posizione sul piano e dalla sua velocità orizz e verticale

Cosa si prende per varianza di uno dell'altro, come è fatto lo stato (vettore i 4 dimensioni di cui 2 posizione xy etc..)
