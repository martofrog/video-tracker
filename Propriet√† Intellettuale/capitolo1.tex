\chapter{Proprietà Intellettuale e Industriale}
cosa sono , differenze etc
\section{Diritto d'autore e Copyright}
breve excursus su tutti e due. 1 pagina e mezzo massimo.
\section{Marchi}
qui bisogna spiegarlo x bene
\section{Brevetti}
Il brevetto è lo strumento giuridico che conferisce all'autore di un'invenzione il monopolio temporaneo di sfruttamento dell'invenzione stessa, ossia il diritto di escludere terzi dall'attuare l'invenzione e dal trarne profitto. 

L'invenzione è la forma di protezione più forte che viene concessa a quei trovati che hanno un alto grado di innovazione, ma che, soprattutto, rappresentano una soluzione nuova ed originale ad un problema tecnico.

Il brevetto rappresenta pertanto un monopolio legale, se pur limitato territorialmente e temporalmente. Tale monopolio legale si giustifica con il fatto che il sistema brevettuale è basato su una forma di scambio: il titolare del brevetto riceve protezione per la propria invenzione e in cambio è obbligato a svelare e a descrivere l'invenzione. Le domande di brevetto e i brevetti già concessi sono infatti pubblicati dagli uffici brevetti di tutto il mondo e ciò li rende una primaria fonte di informazione tecnico-scientifica.

Possono costituire oggetto di brevetto i prodotti, i procedimenti produttivi, le varietà vegetali, mentre non sono brevettabili (art. 45 C.P.I.) ``le scoperte, le teorie scientifiche, i metodi matematici, i piani, i principi ed i metodi per attività intellettuale, per gioco o per attività commerciali, i programmi di elaboratori, le presentazioni di informazioni'' in quanto tali. 

Al di là della statica definizione legislativa riuscire a comprendere che cosa possa essere brevettabile come invenzione, richiede molto studio e molta pratica, anche se in modo sintetico si è soliti dire, con una definizione che soddisfa ben poco, che l'invenzione rappresenta una soluzione innovativa ad un problema tecnico; essendo solamente l'idea di fondo del sistema dei brevetti la stessa in tutti i paesi del mondo, si evidenziano profonde differenze nei vari sistemi brevettuali nazionali e continentali, che vanno non solo ad incidere nelle tecniche di brevettazione, ma discriminano anche nell'insieme delle tipologie di invenzioni brevettabili.

Lasciando l'analisi ancora in superficie rispetto alle casistiche particolari delle varie legislazioni, risulta importante chiarire secondo quali requisiti una invenzione è catalogabile come ``brevettabile'':
\begin{description}
 \item[Requisito di novità] L'oggetto del brevetto deve essere nuovo in modo assoluto, cioè non essere mai stato prodotto o brevettato in nessuna parte del mondo. Il concetto di novità viene inteso in senso ampio e si ricomprende nello "stato della tecnica" tutto ciò che è stato reso pubblico, in Italia o all’estero, prima della data di deposito della domanda di brevetto. Risulta chiarificatore un esempio banale per distinguere la brevettabilità dalla possibilità di produrre e/o sfruttare un invenzione: se un oggetto è stato realizzato o brevettato, ad esempio, in Cina ma non in Italia, ciò significa che chiunque in Italia potrà produrlo e venderlo, ma non certo che possa anche brevettarlo: la differenza è evidente, in quanto senza brevetto potrà agire in regime di libera concorrenza, senza pretendere di avere alcun monopolio.
\item [Requisito di originalità] Chiamato anche ``attività inventiva'' o ``non ovvietà'' sussiste ogni volta che l'invenzione non risulta in modo evidente dallo stato della tecnica per una persona esperta del ramo. Stabilire quando un trovato soddisfi questo requisito è estremamente difficoltoso in quanto è richiesto che l'invenzione per essere brevettabile non debba essere banale, ma rappresentare un progresso, un passo in avanti ``non ovvio'' rispetto allo stato della tecnica attuale. Proprio per stabilire quanto appena detto è spesso interessata la giurisprudenza, anche questa molto altalenante sui giudizi sull’argomento, ed è spesso intorno a questo punto che si giocano le cause relative alla nullità di un brevetto. 
\item [Requisito di industrialità]Risultano brevettabili solo soluzioni che possono essere riprodotte a livello industriale, escludendo tutte le applicazioni artigianali o comunque legate ad un contributo rilevante della persona che le ha realizzate.
\item [Requisito di liceità]Non sono brevettabili invece oggetti che possono ledere il senso del buon costume o essere contrarie all'ordine pubblico, concetti questi in continua evoluzione.
 \end{description}
% 	\subsection{Royalties}
\subsection{Il brevetto in Italia}
\subsection{Il brevetto nell'Unione Europea}
\subsection{Il brevetto negli Stati Uniti}
\section{Analogie e Differenze}
Questa sezione piu che lunga deve essere efficace e sintetica. magari con una tabellina
