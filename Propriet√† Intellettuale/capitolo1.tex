\chapter{Proprietà Intellettuale e Industriale}
cosa sono , differenze etc
\section{Diritto d'autore e Copyright}
breve excursus su tutti e due. 1 pagina e mezzo massimo.
\section{Marchi}
qui bisogna spiegarlo x bene
\section{Brevetti}
Il brevetto è lo strumento giuridico che conferisce all'autore di un'invenzione il monopolio temporaneo di sfruttamento dell'invenzione stessa, ossia il diritto di escludere terzi dall'attuare l'invenzione e dal trarne profitto. 

L'invenzione è la forma di protezione più forte che viene concessa a quei trovati che hanno un alto grado di innovazione, ma che, soprattutto, rappresentano una soluzione nuova ed originale ad un problema tecnico.

Il brevetto rappresenta pertanto un monopolio legale, se pur limitato territorialmente e temporalmente. Tale monopolio legale si giustifica con il fatto che il sistema brevettuale è basato su una forma di scambio: il titolare del brevetto riceve protezione per la propria invenzione e in cambio è obbligato a svelare e a descrivere l'invenzione. Le domande di brevetto e i brevetti già concessi sono infatti pubblicati dagli uffici brevetti di tutto il mondo e ciò li rende una primaria fonte di informazione tecnico-scientifica

Nonostante l'idea di fondo del sistema dei brevetti sia la stessa in tutti i paesi del mondo, esistono profonde differenze nei vari sistemi brevettuali nazionali e continentali, che vanno non solo ad incidere nelle tecniche di brevettazione, ma discriminano anche nell'insieme delle tipologie di invenzioni brevettabili.

	\subsection{Royalties}

\section{Analogie e Differenze}
Questa sezione piu che lunga deve essere efficace e sintetica. magari con una tabellina
