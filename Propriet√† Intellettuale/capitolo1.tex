\chapter{Proprietà Intellettuale e Industriale}
%cosa sono , differenze etc
La sottile differenza morfologica tra le parole ``Proprietà Intellettuale'' e ``Proprietà Industriale'' pone un inganno sulle reali differenze che sussistono tra esse. Prima di confrontare questi due ambiti si cerca di dare una definizione formale dei due concetti.

\`E bene chiarire che il sostantivo \textit{proprietà}, che accomune i due concetti, è usato perchè in questo caso vi è facoltà di escludere terzi da qualsiasi uso di un bene che può essere sia materiale che immateriali: tale facoltà è detta diritto di proprietà.

Per Proprietà Intellettuale si intende la protezione e la valorizzazione delle molteplici forme di creatività intellettuale ed artistica come ad esempio opere musicali, grafiche, editoriali etc. Da un punto di vista giuridico, esse coincidono con tutte le forme di espressione umana e intellettuale che vengono tutelate attraverso il diritto d'autore.

Per Proprietà Industriale invece si definisce l'insieme di istituti giuridici che regolano le creazioni intellettuali umane, ma a contenuto tecnologico; in questo senso si intende opere dell'ingegno (invenzioni) ad ambito industriale. Inoltre si pone la questione sul piano di tutela dei segni distintivi dell'impresa che detiene tale proprietà.

Quindi come si può notare alle precedenti definizioni appaiono delle analogie sul concetto di \textit{proprietà} e diritto sul \textit{bene immateriale}, ma ben altri sostanziali differenze si evincono dal piano strutturale in cui operano: la prima opera nella più classica tutela dei diritti e la seconda nella tutela dell'invenzione industriale inserita nel contesto di una remunerativa attività economica. Quindi sono entrambe forme di tutela, però rilegate ad ambiti applicativi molto diversi. Per fare chiarezza in ciò il nuovo assetto normativo di riferimento (cioè il Decreto Legislativo n. 30 del 2005) è stato intitolato “Codice della proprietà industriale”. Esso è testo unico che raccoglie tutte le norme attinenti al campo dei brevetti e dei marchi. \`E bene sottolineare che resta fuori da questa opera di codificazione la normativa sul diritto d'autore, il cui riferimento è ancora la classi ma brillante legge 633 del 1941 con le varie modifiche nel corso degli anni. \\
Adesso quindi così come abbiamo differenziato il concetto di ``Proprietà Intellettuale'' e ``Proprietà Industriale'', soffermiamoci sui differenti diritti acquisiti con le due proprietà e introduciamo rispettivamente nei prossimi capitoli, il diritto d'autore e il titolo di brevetto.

\section{Diritto d'autore e Copyright}
Si sintetizza ora in questa sezione il diritto che si ottiene in maniera automatica da quella che precedentemente è stata definita come ``Proprietà Intellettuale''.
Nelle scienze giuridiche il diritto d'autore è la posizione giuridica soggettiva dell'autore di un'opera dell'ingegno cui i diversi ordinamenti nazionali e diverse convenzioni internazionali, quale la Convenzione di Berna e la legge vigente, riconoscono la facoltà originaria esclusiva di diffusione e sfruttamento della stessa, ed in ogni caso il diritto ad essere indicato come tale anche quando abbia alienato le facoltà di sfruttamento economico.
Infatti come riporta l'art.25 della Legge 633/1941 dal diritto d'autore scaturiscono due facoltà che si riassuomo in:
\begin{itemize}
 \item Diritti Morali, non cedibili e mai cancellabili. Che si suddividono nei seguenti:

\begin{enumerate}
\item diritto alla paternità dell’opera
\item diritto all’integrit` dell’opera
\item diritto di inedito
\item diritto di ritiro dell’ opera per gravi ragioni morali, cio` il diritto di pentimento.
\end{enumerate}

 \item Diritti Patrimoniali o Economici, possono essere ceduti e terminano dopo il settantesimo anno dalla scomparsa dell'autore.
\end{itemize}


In particolare, il diritto d'autore è una figura propria degli ordinamenti di \textit{Civil Law}come ad esempio l'Italia, laddove in quelli di \textit{Common Law} si parla di copyright, ad esempio negli gli Stati Uniti d'America.
Per quanto riguarda il copyright vi è da dire che giustamente nel gergo comune si utilizzano spesso indistintamente le espressioni copyright e diritto d’autore. I concetti infatti rimangono di una certà similarità se non per il diverso  sostrato social-economico in cui sono stati ideati.

Si può generalmente dire che il concetto di diritto d’autore è sostanzialmente più vasto del copyright: questo perchè
la matrice del copyright è di stampo anglo-americana, cioè dei cosidetti sistemi Common Law ed è nato per tutelare l’industria culturale negli States e in questo senso è stato concepito in primis per tutelare l’interesse dell'accordo
soggetto imprenditoriale/autore. 

Il diritto d’autore, partorito in Europa, fa un passo in più: l’attenzione della normativa si sposta verso la sfera dell’autore e non verso la sfera del commericio; in questo senso il diritto d’autore risulta più ampio del copyright, perchè aggiunge anche i cosidetti diritti morali incedibli.


Conludendo, poichè la trattazione è incentrata sulla tutela del diritto del software è importante sottolineare che ai diritti patrimoniali sul software sono state dedicate norme specifiche per quanto riguarda la legislazione italiana, ossia gli artt.64 bis, 64 ter e 64 quarter della Legge 633/1941 che si conformano alla Direttiva 250/1991/CEE: i diritti esclusivi sui programmi per elaboratori (altresì chiamato \textit{software}) comprendono il diritto di effettuare e autorizzare:

\begin{itemize}
\item la riproduzione temporanea o permanente
\item la traduzione, l’adattamento
\item qualsiasi forma di distribuzione al pubblico compreso la locazione
\end{itemize}



\section{Marchi}
Dopo aver sintetizzato estremamente l'idea di base che sta dietro al diritto d'autore/copyright, si viene ad affrontare uno dei due interessanti concetti legati alla sfera della ``Proprietà Industriale''. In questa sezione infatti approfondiremo i marchi \textit{(trademarks)}.
Il trademark è definito come un segno distintivo ed indicativo creato da un individuo o da un impresa che identifica in maniera chiara un tipo di prodotto o servizio nella mente del consumatore: questa particolartà serve per discriminare il prodotto dagli altri. Il marchio è quindi un tipo di proprietà industriale sulla particolarità che mettie in evidena il prodotto/servizio: parole specifiche, nomi propri, disegni, loghi, cifre, suoni, confezioni e anche tonalità cromatiche. Questo ovviamente purchè queste caratteristiche siano atti a distinguere i prodotti o i servizi di un'impresa da quelli delle altre.Nei sistemi Common Law è possibile anche non registrare un marchio, ma conseguentemente sarà possibile solo proteggerlo nella aree geografiche dove viene usato e non oltre. Negli States inoltre quando ci si riferisce ad un marchio registrato che protegge un servizio si usa il termine \textit{service mark}.

Ovviamente i marchi così come i brevetti, trattati nella sezione \ref{sec:brevetti}, devono possedere dei requisiti per essere registrati. Questi sono riassunti nei seguenti punti:
\begin{description}
 \item[Requisito di novità] un marchio non è nuovo se simile o identico ad un marchio anteriore per prodotto/servizi identici o affini.
 \item[Requisito di originalità] il marchio deve essere atto a evidenziare e contraddistinguere il prodotto dell'impresa
 \item[Requisito di lecità] un marchio non deve essere contrario alla legge vigente o all'ordine pubblico e sociale.
 \item[Requisito di verità] un marchio deve trasmettere ai consumatori il vero ambito del prodotto/servizio. Non deve essere ingannevole. 
 \end{description}

Il principale accordo internazione er garantire e facilitare la registrazione di marchi in multiple legislazione risulta l'accordo o protocollo di Madrid. Esso costituisce un sistema centrale di amministrazione per una sicura registrazione dei trademarks estendendo la protezione di una \textit{registrazione internazionale} ottenutra tremie il WIPO \textit{(World Intellectual Property Organization)}. % This international registration is in turn based upon an application or registration obtained by a trade mark applicant in its home jurisdiction.

The primary advantage of the Madrid system is that it allows a trademark owner to obtain trademark protection in many jurisdictions by filing one application in one jurisdiction with one set of fees, and make any changes (eg. changes of name or address) and renew registration across all applicable jurisdictions through a single administrative process. Furthermore, the "coverage" of the international registration may be extended to additional member jurisdictions at any time.

\section{Brevetti} \label{sec:brevetti}
Il brevetto è lo strumento giuridico che conferisce all'autore di un'invenzione il monopolio temporaneo di sfruttamento dell'invenzione stessa, ossia il diritto di escludere terzi dall'attuare l'invenzione e dal trarne profitto. 

L'invenzione è la forma di protezione più forte che viene concessa a quei trovati che hanno un alto grado di innovazione, ma che, soprattutto, rappresentano una soluzione nuova ed originale ad un problema tecnico.

Il brevetto rappresenta pertanto un monopolio legale, se pur limitato territorialmente e temporalmente. Tale monopolio legale si giustifica con il fatto che il sistema brevettuale è basato su una forma di scambio: il titolare del brevetto riceve protezione per la propria invenzione e in cambio è obbligato a svelare e a descrivere l'invenzione. Le domande di brevetto e i brevetti già concessi sono infatti pubblicati dagli uffici brevetti di tutto il mondo e ciò li rende una primaria fonte di informazione tecnico-scientifica.

Possono costituire oggetto di brevetto i prodotti, i procedimenti produttivi, le varietà vegetali, mentre non sono brevettabili (art. 45 C.P.I.) ``le scoperte, le teorie scientifiche, i metodi matematici, i piani, i principi ed i metodi per attività intellettuale, per gioco o per attività commerciali, i programmi di elaboratori, le presentazioni di informazioni'' in quanto tali. 

Al di là della statica definizione legislativa riuscire a comprendere che cosa possa essere brevettabile come invenzione, richiede molto studio e molta pratica, anche se in modo sintetico si è soliti dire, con una definizione che soddisfa ben poco, che l'invenzione rappresenta una soluzione innovativa ad un problema tecnico; essendo solamente l'idea di fondo del sistema dei brevetti la stessa in tutti i paesi del mondo, si evidenziano profonde differenze nei vari sistemi brevettuali nazionali e continentali, che vanno non solo ad incidere nelle tecniche di brevettazione, ma discriminano anche nell'insieme delle tipologie di invenzioni brevettabili.

Lasciando l'analisi ancora in superficie rispetto alle casistiche particolari delle varie legislazioni, risulta importante chiarire secondo quali requisiti una invenzione è catalogabile come ``brevettabile'':
\begin{description}
 \item[Requisito di novità] L'oggetto del brevetto deve essere nuovo in modo assoluto, cioè non essere mai stato prodotto o brevettato in nessuna parte del mondo. Il concetto di novità viene inteso in senso ampio e si ricomprende nello "stato della tecnica" tutto ciò che è stato reso pubblico, in Italia o all’estero, prima della data di deposito della domanda di brevetto. Risulta chiarificatore un esempio banale per distinguere la brevettabilità dalla possibilità di produrre e/o sfruttare un invenzione: se un oggetto è stato realizzato o brevettato, ad esempio, in Cina ma non in Italia, ciò significa che chiunque in Italia potrà produrlo e venderlo, ma non certo che possa anche brevettarlo: la differenza è evidente, in quanto senza brevetto potrà agire in regime di libera concorrenza, senza pretendere di avere alcun monopolio.
\item [Requisito di originalità] Chiamato anche ``attività inventiva'' o ``non ovvietà'' sussiste ogni volta che l'invenzione non risulta in modo evidente dallo stato della tecnica per una persona esperta del ramo. Stabilire quando un trovato soddisfi questo requisito è estremamente difficoltoso in quanto è richiesto che l'invenzione per essere brevettabile non debba essere banale, ma rappresentare un progresso, un passo in avanti ``non ovvio'' rispetto allo stato della tecnica attuale. Proprio per stabilire quanto appena detto è spesso interessata la giurisprudenza, anche questa molto altalenante sui giudizi sull’argomento, ed è spesso intorno a questo punto che si giocano le cause relative alla nullità di un brevetto. 
\item [Requisito di industrialità]Risultano brevettabili solo soluzioni che possono essere riprodotte a livello industriale, escludendo tutte le applicazioni artigianali o comunque legate ad un contributo rilevante della persona che le ha realizzate.
\item [Requisito di liceità]Non sono brevettabili invece oggetti che possono ledere il senso del buon costume o essere contrarie all'ordine pubblico, concetti questi in continua evoluzione.
 \end{description}

Non da trascurare è l'aspetto legato ai diritti che poi scaturiscono dall'invenzione stessa; in seguito all'invenzione scaturiscono nei confronti dell'autore due tipologie diverse di diritto: il diritto morale sull'invenzione ed il diritto di brevetto. Mentre il primo concerne un'area strettamente personale e non è cedibile, il secondo riguarda lo sfruttamento economico dell'invenzione, e quindi risulta cedibile.
% 	\subsection{Royalties}
\subsection{Il brevetto in Italia}
\subsection{Il brevetto nell'Unione Europea}
\subsection{Il brevetto negli Stati Uniti}
\section{Analogie e Differenze}
Questa sezione piu che lunga deve essere efficace e sintetica. magari con una tabellina
