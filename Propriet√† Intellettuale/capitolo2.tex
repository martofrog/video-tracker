\chapter{La proprietà intellettuale in campo software}
Dopo aver delineato la sfera della Proprietà Industriale e Intellettuale, si cerca di fare luce sull'attuale sistema di tutela dei diritti che si acquisicono con la scrittura di un programma per elaboratori; si cerca di cogliere in esso la giusta forma di tutela, confrontando realtà diverse come in UE e in USA. 

\section{Il sistema giuridico delle licenze}
Attualmente il software è concepito come espressione dell'intelletto umano e ovviamente ricade nella tutela della Proprietà Intellettuale. Per questo chiunque scrive un programma per elaboratore detiene il così detto copyright, cioè tutti i diritti che si sono evidenziati nel capitolo 1.\\
Principalmente il software è prodotto con enormi investimenti per uno scopo molto semplice: l'uso. Negli anni '80 e '90, le software house hanno lucrato su questa rivoluzione digitale fino a diventare multinazionali o comunque colossi dell'informatica.\\                                                      
Si è cercato fin da subito quindi di tutelare il software in maniera tale da avere la possiblità di venderlo, pur rimanendo sempre i proprietari. Come si sa su di esso si possiede il diritto d'autore (cioè la proprietà intellettuale, non la proprietà del supporto con cui viene venduto) ed è proprio su questa facoltà che è nato il contratto tra un licenziatario \footnote{colui che ne detiene il copyright e ne cede alcuni in cambio di denaro o altro} e un licenziante \footnote{qualsiasi utente dell'opera}. Questo accordo scritto prende comunemente il nome di licenza.
Nel particolare la licenza segue il modello uno a molti tra licenziante e licenziatario e si definisce con il termine ``contratto di licenza''. La parola contratto deriva dal modo di tutelare la trasmissione dei diritti d'autore cioè nel senso classico di accordo  tra due o più parti per costituire, regolare o estinguere un rapporto giuridico patrimoniale”.
La termine licenza invece si definisce come un atto unilaterale giuridico, originario del diritto amministrativo, con cui un soggetto concede un' autorizzazione a compiere una determinata attività. \`E importante sottolineare come l'uso del vocabolo ``licenza'' derivi dal termine coniato negli States e tradotto letteramente da \textit{mass market licenses of copyright material }.
In sintesi quindi l'obietto che si prefigge una licenza riguarda cessione di alcuni diritti come, ad esempio, il diritto all'uso: infatti con l'acquisto di un software, e la conseguente sottoscrittura alla licenza, non si compra il software in sè, inalienabile dall'autore, ma il diritti all'uso e in alcuni casi (come nel copyleft) anche il diritto alla modifica e alla copia/ridistribuzione.

\section{La situazione dell'UE sui brevetti software}
La situazione europea e mondiale sulla questione dei brevetti software è molto spinosa, in quanto risulta molto complesso delimitare i confini per cui il software inteso come esclusivo codice possa ritenersi brevettabile oppure una invenzione tecnica basata su software possa ritenersi non brevettabile.

Nell'Unione Europea l'Organizzazione europea dei Brevetti ha rilasciato molti brevetti su invenzioni basate almeno in parte su software da quando è in vigore, dagli anni '70 è in vigore la Convenzione europea dei Brevetti, nonostante proprio in Europa, per la precisione in Francia nel 1968, sia nata la prima norma in materia brevettuale tesa ad escludere dalla tutela i programmi per elaboratore.

L'articolo 52  della convenzione esclude esplicitamente i programmi per computer dalla brevettabilità (comma 2), intesi come programmi per computer in quanto tali (comma 3). L'interpretazione data all'articolo è che ogni invenzione che offre un contributo tecnico non ovvio o risolve problemi tecnici in maniera non banale è brevettabile anche se comprende una parte software.

Ovviamente non può essere sufficiente affidare l'intera legislazione di un settore così importante a livello scientifico ed economico ad una mera distinzione di carattere semantico, ma deve essere approfondito l'ambito di utilizzo e di realizzazione del software stesso. 

Prima di tutto è da notare che è impossibile che il legislatore abbia impiegato la definizione ``programmi in quanto tali'' riferendosi esclusivamente alla forma codificata delle informazioni, lasciando brevettabili tutti i processi che le istruzioni producono in quanto il software è stato incluso nella categria delle opere non brevettabili come ``attività intellettuale'' e non come ``presentazione di informazioni''.

Per inquadrare correttamente il software all'interno della disciplina brevettuale occorre prendere in considerazione due momenti: innanzitutto come registrato su un supporto di memorizzazione (inteso quindi come insieme di informazioni, sequenza di istruzioni), in secondo luogo all'atto dell'esecuzione, quando tramite le istruzioni va a regolare il comportamento della macchina. 

Il primo momento dei due appena descritti è stabilmente appartenente all'area del non-brevettabile, essendo il funzionamento del supporto completamente indipendente da ciò che ne viene memorizzato sopra, ed essendo totalmente libero il genere di informazioni che possono essere registrate sul supporto. Ristretto a questo momento il concetto di software è riconducibile alle ``presentazioni di informazioni'', escluse dalla brevettabilità con chiarezza (articoli 52 n.2 lett.\textit{d} CBE e 12 co.2 lett.\textit{c} l.i.).

Il momento di interesse alla trattazione è quello che riguarda il funzionamento del software all'interno dell'elaboratore, in quanto si realizza lo scopo pratico per cui il software è stato scritto.

---> Qui c'è da mettere in pratica il punto 7 (pag 29) e la roba sull'attualità, che è al link http://www.fsfeurope.org/projects/swpat/status.it.html poi il paragrafo è finito, diciamo che il materiale c'è tutto, va scritto ma non è un problema!
\section{I brevetti software in USA}
Molto diversa dalla situazione europea sui brevetti del software appare la gestione dei brevetti software negli Stati Uniti.  





