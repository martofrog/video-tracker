\chapter{La proprietà intellettuale in campo software}
Dopo aver delineato la sfera della Proprietà Industriale e Intellettuale, si cerca di fare luce sull'attuale sistema di tutela dei diritti che si acquisicono con la scrittura di un programma per elaboratori; si cerca di cogliere in esso la giusta forma di tutela, confrontando realtà diverse come in UE e in USA. 

\section{Il sistema giuridico delle licenze}
Attualmente il software è concepito come espressione dell'intelletto umano e ovviamente ricade nella tutela della Proprietà Intellettuale. Per questo chiunque scrive un programma per elaboratore detiene il così detto copyright, cioè tutti i diritti che si sono evidenziati nel capitolo 1.\\
Principalmente il software è prodotto con enormi investimenti per uno scopo molto semplice: l'uso. Negli anni '80 e '90, imprese come la famosa Microsoft (e non solo) hanno lucrato su questa rivoluzione digitale fino a diventare multinazionali o comunque colossi dell'informatica.\\
Si è cercato fin da subito quindi di tutelare il software in maniera tale da avere la possiblità di venderlo, pur rimanendo sempre i proprietari. Come si sa su di esso si possiede il diritto d'autore (cioè la proprietà intellettuale, non la proprietà del supporto con cui viene venduto) ed è proprio su questa facoltà che è nato il contratto tra un licenziatario \footnote{colui che ne detiene il copyright} e un licenziante \footnote{qualsiasi utente dell'opera}. Questo accordo scritto prende comunemente il nome di licenza.
Nel particolare la licenza segue il modello uno a molti licenziante-licenziatario e il concetto di ``contratto di licenza'' si definisce come un atto unilaterale giuridico, originario del diritto amministrativo, con cui un soggetto concede un' autorizzazione a compiere una determinata attività.
Ovviamente l'attività con cui ci si riferisce in questo caso è quella che deriva dalla cessione di alcuni diritti come, ad esempio, il diritto all'uso: infatti con l'acquisto di un software non si compra il software in sè, inalienabile dall'autore, ma si comprano i diritti all'uso e in alcuni casi (come nel copyleft) anche il diritto alla modifica e alla copia/ridistribuzione.

\section{La situazione dell' UE sui brevetti software}

\section{I brevetti software in USA}
