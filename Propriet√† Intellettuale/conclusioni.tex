 \chapter{Conclusioni}

Con questo elaborato si è voluto differenziare nel primo capitolo i concetti di Proprietà Intellettuale e Industriale, che pur rimanendo molto simili, nascono da un sostrato abbastanza diverso, anche se rispettivamente la prima copre una serie di diritti maggiore della seconda, come riportato nella visione insiemistica in figura \ref{fig:PI} . I tipi di tutela quindi che derivano da entrambi sono leggermente diversi in quanto il primo si attaglia al diritto d'autore in generale e l'altro alla tutela dell'invenzione come prodotto industriale. Fatto questo si è passati a definire brevemente il concetto di copyright, dilungandosi invece leggermente nella descrizione dei brevetti e marchi, forme di tutela della Proprietà Intellettuale, in quanto più coerenti con la traccia della trattazione.

Il capitolo secondo si contraddistingue perché, dopo aver data la definizione di brevetto, entra nel merito di tutta la trattazione: la tutela del software attraverso lo strumento del brevetto. Su questo concetto emergono tutt'oggi pareri contrastanti, per questo motivo la trattazione si è fatta molto interessante. Si è percorso un breve excursus sulla tutela giuridica del software secondo copyright, successivamente si è riassunto la storia dei brevetti software in UE e in USA, dandone una visione dall'alto e riportando i casi giuridici principali di ogni paese.

La trattazione poi si è spostata su tempi più recenti e ambiti tecnici. In particolare nel capitolo terzo si è osservato il tagliente punto di vista della Free Software Foundation nei confronti dei software patens, legati anche ad altre tecnologie recenti di restrizione dei diritti degli utenti. Si è quindi analizzato i nuovi articoli che compongono la nuova licenza copyleft: la GNU General Public License versione 3.
Infine si è dato un esempio degli effetti provati da questa, nella distribuzione del software gplv3 con software house che hanno sottoscritto patents agreement con altre lampante in questo è il caso Novell-Microsoft e il passaggio di licenza del software di interoperabilità tra reti Samba.

Ricordando di dare una veste pratica oltre che legale, l'ultimo capitolo prende come esempio il brevetto del formato di compressione audio per antonomasia, cioè l'mp3: il brevetto non è unico, ma sono una serie di brevetti detenuti da molte società; è stato quindi effettuato un lungo lavoro per districarsi nella rete brevettuale che si è formata. Prendendo questo caso, si vuole mostrare quali sono gli effetti della brevettazione, della difficoltà legata alla eterogeneità delle varie legislazioni e degli effetti apportate al free software o a software house commerciali.

Dalla trattazione è quindi emerso che il software è giusto che venga tutelato dal diritto d'autore; è anche vero d'altro canto che non può essere concepito solo come una mera composizione logico-intellettuale, visto l'enorme commercio del mercato del software. E quindi in questo senso è giusto affermare che anche il software,pur rimanendo un bene immateriale diverso dall'hardware, è entrato nella schiera dei prodotti industriali a tutti gli effetti. Appurato questo è giusto chiedersi se è lecito lasciare il software sotto copyright o addirittura applicare la nozione di brevetto, concepita per prodotti fisici o processi industriali, anche agli algoritmi implementati. Oppure invece notando l'uso che poi ne viene fatto dalle software house, potrebbe essere possibile negare l'applicazione del brevetto al software, in maniera congrua con quanto affermato dalla Free Software Foundation nella GPLv3, magari trovando una via alternativa tra la tutela intellettuale troppo generica e la tutela industriale troppo legata al monopolio.
